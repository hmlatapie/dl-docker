\documentclass{article}

\usepackage{teaching, array, hyperref}

\begin{document}

\begin{tdoc}{CHEM 116}{Unit 1, Lecture 1}{Numerical Methods and Statistics} 

\section{Course Structure}

\subsection{Goal of course}

Real world problems aren't perfect, they require statistics. Real
world problems require computers. This course will equip you for that.

\subsection{Components of Course}

{\bf Probability Theory}: The analysis of randomness of sets and variables.\vspace{0.2cm}\\
{\bf Statistics}: The analysis of data using probability theory\vspace{0.2cm}\\
{\bf Numerical Methods}: The solution of numerical problems with algorithms\vspace{0.2cm}\\

\subsection{Course Textbooks}

\begin{description}

\item[Principles of Statistics]{A book concerned with the theory of statistics. It has hilarious examples and seems to be set in the Charles Dickens universe}

\item[Practical Statistics Explained]{A practical handbook concerned with the application of statistics}

\item[Code Academy]{A simple introduction to python: \href{https://www.codecademy.com/learn/python}{www.codecademy.com/learn/python}}

\item[Python in Easy Steps]{Accessible book covering Python basics. Bright colors, friendly cartoons that talk to you in the margins.}

\end{description}

\subsection{Grading}

This course is primarily electronic so homework and projects are more
heavily weighted.

\subsection{Policies}
See syllabus

\subsection{Example problems}

\begin{itemize}

\item Flow on graphs (electricity, chemical reactions, fluids)
\item Fitting data to kinetic rate laws
\item Solving equations numerically ($\sin(x) = x$)
\item Create a website containing equations, graphs and text for distributing work

\end{itemize}

\subsection{Projects}

Will be discussed more in the future. 

\subsection{Python}

\begin{enumerate}

\item Good introductory language, used by 8/10 top CS departments
\item Wide use out of industry
\item Less used than MatLab and Excel, but growing
\item Many libraries in engineering and significantly more than MatLab outside of engineering
\item These libraries allow us to mix cool things together, like live websites with instrument data
\item Programs will be easy distribute, easy to install, easy to view by people with no programming experience

\end{enumerate}

\section{Probability Theory}

\subsection{Sample spaces}

Sets, ordered sets, integers, real numbers. 

{\bf Examples:}
\begin{enumerate}

\item $\{A, B\}$ (set)
\item The roll of a die: $1-6$ (integers, which is an ordered set, which is a set)
\item Cards in a 52 card deck (ordered set, which is a set)
\item All possible molecules formed in a chemical reaction (set)
\item Electronic configurations of a molecule (set, possibly ordered with energy)
\item Flow rate into a tank (real number)
\item Temperature (real number)
\item Value of cryptographic key (integer)
\item Pet type owned (cat, dog, fox) (set)

\end{enumerate}

Think about what is ordered, what is a set, what are real numbers, etc

\subsection{Probability of sample spaces}
Probability assigns a number $P$ to each sample $x$. The only
requirement is that the sum of $P(x)$ over the sample space is
$1$. That condition is called normalization.

Notice that even if $\sum_{X} P(x) > 1$, as long as it's finite we can
\emph{normalize} the probability by dividing it by a constant such
that its sum is $1$. Example: 
\begin{equation}
P(\textrm{die roll}) = \textrm{value of die}
\end{equation}

is not normalized, because $\sum_X P(x) = 21$. Normalizing it gives:

\begin{equation}
P(\textrm{die roll}) = \frac{\textrm{value of die}}{21}
\end{equation}

\subsection{Probability Algebra - For Samples}

{\bf OR}\vspace{0.5cm}\\
The probability of sample A or sample B being drawn is exactly:

\begin{equation}
P(A\,\textrm{or}\,B) = P(A) + P(B)
\end{equation}
\vspace{0.2cm}

{\bf AND}\vspace{0.5cm}\\

If we draw two samples sequentially (!) and independently (!):
\begin{equation}
P(A\,\textrm{and}\,B) = P(A)P(B)
\end{equation}

For now, independence between trials means the outcome of one doesn't
affect the probability of the other.

\vspace{0.2cm}

{\bf NOT}\vspace{0.5cm}\\

\begin{equation}
P(\sim A) = 1 - P(A)
\end{equation}
\vspace{0.2cm}

These statements allow us to bridge probabilities together:\vspace{0.2cm}\\

\texttt{Draw an ace of spades AND roll a 2 OR roll a 4 AND NOT own a cat}\\

\vspace{0.2cm}Notice that AND is used to bridge together independent samples, OR is
used to bridge together multiple possibilities, and NOT is used to
``invert'' probabilities.\vspace{0.2cm}

What is wrong with this statement? \vspace{0.2cm}\\

\texttt{Draw an ace of spades OR roll a 2}\vspace{0.2cm}\\

You cannot join samples which are from different sample spaces

\subsection{Events}

Events are collections of samples occurring. Events can overlap, occupy
entire sample space or be generally messy.

\begin{itemize}

\item Roll an odd number
\item Have a flow greater than 2 kg/s
\item Have a flow between 2 and 4 kg/s

\end{itemize}
\vspace{0.5cm}

{\bf AND} and {\bf NOT} apply to probabilities of events. {\bf OR}
only applies if the events do not overlap (mutually exclusive). For
example if event A is roll an odd number and event B is roll a 3,
event A includes event B meaning the normal {\bf OR} does not
apply. To deal with this, generally you redefine your event or
compute the intersection $P(A\cap B)$ and use:
\begin{equation}
P(A\,\textrm{or}\,B) = P(A) + P(B) - P(A \cap B)
\end{equation}
I hesitate to say $P(A \cap B)$ means probability of $A$ and $B$
occurring, because that sounds similar to what was above. Think
instead that $P(A \cap B)$ is the probability of $A$ and $B$ occurring
simultaneously. Alternatively, $P(A \cap B)$ is the sum of
probabilities of the elements in the sample space that overlap in $A$
and $B$.

\subsection{Independent samples and combination vs permutation}

When sampling multiple independent trials/events/samples, the {\bf
  AND} rule applies to permutations. A {\bf combination} of events
have no ordering - order does not matter. A particular sequence from
those events is called a {\bf permutation} - order
matters\vspace{0.2cm}

Sometimes we do not care about order, like when rolling two dice:
$3,2$ is the same as $2,3$. There are two permutations, and we must
consider both to get the probability of the whole 2 observation
combination. This can be done with the {\bf OR} rule:

\begin{equation}
P(3,2) = P(3)\times P(2) + P(2) \times P(3) = 2\times P(3)\times P(2)
\end{equation}

So, to get the probability of a permutation we can just use the {\bf
  AND} rule. To get the probability of a combination, we must use the
{\bf AND} rule for each permutation and use the {\bf OR} rule to
combine each permutation that is possible for the combination. For
example, a combination of 1,2,3 is the sum of the probabilities of
each possible permutation of rolling a 1,2,3:
\begin{enumerate}
\item $P(\textrm{perm}: 1,2,3) = P(1)\cdot P(2) \cdot P(3) = \frac{1}{6^3}$
\item $P(\textrm{perm}: 1,3,2) = P(1)\cdot P(2) \cdot P(3) = \frac{1}{6^3}$  
\item $P(\textrm{perm}: 2,3,1) = P(1)\cdot P(2) \cdot P(3) = \frac{1}{6^3}$
\item $P(\textrm{perm}: 2,1,3) = P(1)\cdot P(2) \cdot P(3) = \frac{1}{6^3}$
\item $P(\textrm{perm}: 3,1,2) = P(1)\cdot P(2) \cdot P(3) = \frac{1}{6^3}$
\item $P(\textrm{perm}: 3,2,1) = P(1)\cdot P(2) \cdot P(3) = \frac{1}{6^3}$
\end{enumerate}
So the probability of the combination of 1,2,3 is
$\frac{6}{6^3}$. 

\subsection{Uniform Probability Sample Spaces} 

If all samples have equal probabilities probabilities, the probability of a
sample is

\begin{equation}
P(x) = \frac{1}{Q}
\end{equation}
where $Q$ is the sample space size.
\vspace{0.2cm}


The probability of an \emph{event} occurring is the number of samples
in the event, $q$, divided by the size of the sample space, $Q$:

\begin{equation}
P(\textrm{event}) = \frac{q}{Q}
\end{equation}

For example, the probability of rolling an odd number is $3 / 6$. 


The probability of a \emph{combination} occurring is then the number of
permutations, $n$, times the probability of a single permutation:
\begin{equation}
P(\textrm{combination}) = n P(\textrm{permutation})
\end{equation}
\vspace{0.2cm}

See the above permutation example to see this. Notice this is about multiple observations, whereas the previous two equations are about single observations.

\subsection{Tricky concepts to review}

\begin{description}

\item [Independence]: For now it means the trials don't affect one
another. One way to tell is if the trials can be permuted, they are
independent.\vspace{0.2cm}\\

\item[{\bf OR}]: Cannot combine statements in different samples spaces, whereas {\bf AND} can cross sample spaces.\vspace{0.2cm}\\

\item[Normalization]: As long as your made-up probability measure is
finite everywhere, it can be normalized.

\item[Combination vs Permutation]: A permutation is a particular
  ordering of a combination. 

\item[Event vs. Observation]: I've used observation here to indicate
  the generation of a sample (outcome). The generation of the sample
  may correspond to an event, but don't confuse the word `event' with
  meaning we generated a sample. An event is a set of samples, where
  if any elements of the set occur then we say the event occured. An
  example of an event is the set of ${1,5}$ for rolling a die. The
  observation, or generation of a sample, could be 1 for example.

\end{description}

\end{tdoc}

\end{document}
